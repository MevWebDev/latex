\documentclass[polski]{article}
\usepackage[utf8]{inputenc}
\usepackage[T1]{fontenc}
\usepackage{babel}
\usepackage{lmodern} % Lepsze wsparcie dla fontów T1
\usepackage{graphicx}
\usepackage{lipsum}

\title{Tytuł Twojej Pracy}
\author{Imię Nazwisko}
\date{\today}

\begin{document}

\maketitle

\begin{abstract}
    To jest streszczenie pracy. \lipsum[1]
\end{abstract}

\tableofcontents

\section{Wprowadzenie}
Wprowadzenie do tematu.

\section{Pierwszy Rozdział}
\subsection{Sekcja 1.1}
Tekst pierwszej sekcji. \textbf{Pogrubienie}, \textit{Kursywa}.

\subsection{Sekcja 1.2}
Inny tekst w pierwszym rozdziale.

\section{Drugi Rozdział}
\subsection{Sekcja 2.1}
Tekst drugiej sekcji. Przykład trybu matematycznego: $a^2 + b^2 = c^2$.

\section{Rysunki}
\begin{figure}[h]
    \centering
    \includegraphics[width=0.5\linewidth]{rysunek1.jpg}
    \caption{Podpis do pierwszego rysunku.}
    \label{sec:rysunek1}
\end{figure}

\begin{figure}[h]
    \centering
    \begin{tabular}{c}
        \includegraphics[width=0.4\linewidth]{rysunek2.jpg} \\
        \textbf{Rysunek skomponowany z tekstem.}
    \end{tabular}
    \caption{Podpis do drugiego rysunku.}
    \label{sec:rysunek2}
\end{figure}

\section{Tabela}
\begin{table}[h]
    \centering
    \begin{tabular}{|c|c|}
        \hline
        Kolumna 1 & Kolumna 2 \\
        \hline
        Wartość 1 & Wartość 2 \\
        Wartość 3 & Wartość 4 \\
        \hline
    \end{tabular}
    \caption{Tabela odnosząca się do tekstu.}
    \label{sec:tabela}
\end{table}

\section{Odniesienia}
W sekcji \ref{sec:rysunek1} znajduje się opis pierwszego rysunku, a w sekcji \ref{sec:tabela} jest tabela.

\section{Podsumowanie}
Podsumowanie pracy.

\begin{thebibliography}{99}
    \bibitem{przyklad1} Autor, \emph{Tytuł Pracy}, Wydawnictwo, Rok.
    \bibitem{przyklad2} Inny Autor, \emph{Inny Tytuł}, Inne Wydawnictwo, Inny Rok.
\end{thebibliography}

\end{document}
